%-------------------------------------------------------------------------------
%	SECTION TITLE
%-------------------------------------------------------------------------------
\cvsection{Experience}


%-------------------------------------------------------------------------------
%	CONTENT
%-------------------------------------------------------------------------------
\begin{cventries}

%---------------------------------------------------------
  \cventry
    {Staff Data Science Engineer (promoted in 2020 and 2022)} % Job title
    {Indigo Agriculture} % Organization
    {Boston, MA} % Location
    {April 2022 -} % Date(s)
    {
      \begin{cvitems} % Description(s) of tasks/responsibilities
        \item {
            Built an agricultural data validation and imputation engine 
            to power the monetization of sustainable agriculture. 
            Developed validation and imputation methodologies to 
            support the measuring, reporting, and verification of 
            farming activity (38,500 fields, $>$2.5M acres or $\sim$2.5x the
            area of Rhode Island). Worked with soil 
            scientists, data scientists, and product owners / managers 
            to integrate software in a pipeline that generates 
            verified carbon credits from organic carbon sequestered in 
            agricultural soils. Carbon program achieved 
            5x and 3x year-over-year growth in its second 
            and third credit issuances with monotonically increasing conversion 
            rates. This work resulted in 
            \href{https://patents.google.com/patent/US20230186408A1}{multiple} 
            \href{https://patents.google.com/patent/US20230078852A1}{patents}.
        }
        \item {
            Acted as team lead on a new squad that developed the carbon credit generation pipeline
            for one year. Worked with squad to define intermediate and short-term work planning processes.
            Mentored junior engineers and fostered a supportive, inclusive environment with space for
            team member growth.
        }
        \item {
            Maintained, updated, and documented a distributed (AWS Lambda, Batch) service 
            that summarizes remote sensing observations over field geometries. Service scaled elastically
            to meet bursty loads, processing jobs with variable, high concurrency. The service supported
            arbitrary calculations across imagery bands (e.g vegetative indices from RGB and NIR bands of
            \href{https://www.esa.int/Applications/Observing_the_Earth/Copernicus/Sentinel-2}{Sentinel-2})
            and produced spatial summary statistics, yielding time series collated from hundreds of
            source images per area of interest. Service integrated with 25+ public imagery products.
        }
        \item {
            Designed and built an API for providing inferred agronomic activity for user-defined areas of interest.
            The API served machine learning inferences and public reference data. Inferences were used for data entry
            and validation in the user-facing front end of multiple sustainable agriculture programs.
            Built the infrastructure to orchestrate model inference across many unique machine learning model cohorts,
            persist the results, and gather the predictions into easily consumable API responses.
        }
      \end{cvitems}
    }

%---------------------------------------------------------
  \cventry
    {Algorithm Engineer} % Job title
    {Climacell} % Organization
    {Boston, MA} % Location
    {Sep. 2018 - Apr. 2019} % Date(s)
    {
      \begin{cvitems} % Description(s) of tasks/responsibilities
        \item {
            Developed a neural network-based classifier that detected
            precipitation in nearly real time using cellular data link attenuation
            time series. Gathered and cleaned historical ground truth 
            data from weather stations to train production models.
        }
        \item {
            Built a feature generation and prediction pipeline that integrated
            with the cellular link time series ingest pipeline in production,
            publishing low-latency precipitation inferences at tens of thousands
            of locations.
        }
      \end{cvitems}
    }

%---------------------------------------------------------
  \cventry
    {Ph.D. Research} % Job title
    {Massachusetts Institute of Technology} % Organization
    {Cambridge, MA} % Location
    {Aug. 2012 - May 2018} % Date(s)
    {
      \begin{cvitems} % Description(s) of tasks/responsibilities
        \item {
            Researched the fundamental physics of nanostructured
            semiconductor materials, relevant to energy conversion,
            transistors, and light emitting applications. Designed
            and performed experiments and analysis culminating
            in a
            \href{https://dspace.mit.edu/handle/1721.1/118263}{Ph.D. thesis},
            \href{https://scholar.google.com/citations?user=5-TA9cAAAAAJ}{11 peer reviewed papers},
            and a \href{https://patents.google.com/patent/US10855046B2/en}{patent}.
        }
        \item {
            Took a course in scientific communication led by
            experts from the Boston Museum of Science focused on
            effective oral, written, and visual communication.
            The course focused on communicating complex topics
            to diverse audiences. Received conference awards for oral (research talk) and written/visual (poster) communication.
        }
      \end{cvitems}
    }
\end{cventries}
